\documentclass[journal]{IEEEtran}


\usepackage{capt-of}
\usepackage{placeins}
%\usepackage[toc,page]{appendix}
\usepackage{multirow}
\usepackage{xspace}
\usepackage{latexsym}
\usepackage{amssymb}
\usepackage{listings}
\usepackage{tikz}

\usepackage{color}
\usepackage[T1]{fontenc}

\definecolor{dkgreen}{rgb}{0,0.6,0}
\definecolor{gray}{rgb}{0.5,0.5,0.5}
\definecolor{mauve}{rgb}{0.58,0,0.82}
\definecolor{orange}{rgb}{1.0,0.49,0.0}

\usepackage{enumitem}
\usepackage{pifont}

\lstset{frame=single,
  language={Java},
  aboveskip=3mm,
  belowskip=3mm,
  showstringspaces=false,
  basicstyle={\small\ttfamily},
  numbers=left,
  stepnumber=1,
  numberstyle=\tiny\color{gray},
  keywordstyle=\color{orange},
  commentstyle=\color{dkgreen},
  breaklines=true,
  breakatwhitespace=true,
  tabsize=1,
  morecomment=[l]{\\*},
  morekeywords={}
}
%\usepackage[retainorgcmds]{IEEEtrantools}
%\usepackage{bibentry}  
\usepackage{xcolor,soul,framed} %,caption

\colorlet{shadecolor}{yellow}
% \usepackage{color,soul}
%\usepackage[pdftex]{graphicx}
%\graphicspath{{../pdf/}{../jpeg/}}

%\DeclareGraphicsExtensions{.pdf,.jpeg,.png}

\usepackage[cmex10]{amsmath}
%Mathabx do not work on ScribTex => Removed
%\usepackage{mathabx}
\usepackage{array}
\usepackage{mdwmath}
\usepackage{mdwtab}
\usepackage{eqparbox}
\usepackage{url}

\newcommand\tab[1][1cm]{\hspace*{#1}}
\renewcommand{\floatpagefraction}{.99}
\renewcommand{\textfraction}{.01}

\newcommand{\tlaplus}{TLA\textsuperscript{+}\xspace}
\newcommand{\EXCEPT}{\textsc{except}}
\newcommand{\IF}{\textsc{if}}
\newcommand{\THEN}{\textsc{then}}
\newcommand{\ELSE}{\textsc{else}}
\newcommand{\seq}[1]{\langle #1 \rangle}

\newcommand{\keyword}[1]{\textbf{#1}}
\newcommand{\entity}[1]{\ensuremath{\langle}#1\ensuremath{\rangle}}


%\bstctlcite{IEEE:BSTcontrol}


%=== TITLE & AUTHORS ====================================================================
\begin{document}
\bstctlcite{IEEEexample:BSTcontrol}
    \title{Formal Verification of Distributed Algorithms using Distributed PlusCal}
  \author{Stephan Merz, Horatui Cirstea, Heba Al-kayed \\~\IEEEmembership{Inria, Loria}
% <-this % stops a space
}  

% ====================================================================
\maketitle

\IEEEpeerreviewmaketitle

\section{Motivations}

\IEEEPARstart{D}{istributed} systems are based on continuous interactions among components, these interactions produce bugs that are difficult to find by testing as they tend to be non-reproducible or not covered by test-cases.

Formal verification methods like model checking \cite{ModelCheckingTLA} depend on the users ability to specify the concerned system, \tlaplus \cite{tlaplus} is a formal language used to describe algorithms, it provides a flexibility and an expressiveness that enables it to specify and verify complicated algorithms concisely.

\tlaplus relies on mathematical logic and formulas for structuring specifications whereas The PlusCal language \cite{pcalAlgo} provides a simple pseudo-code like interface for the user to express concurrent systems. It maintains the expressiveness of \tlaplus while providing the user with a more familiar syntax.
Although PlusCal is very useful, when it comes to distributed algorithms it may enforce some limitation that make it difficult to express them in a natural way. As an example, distributed algorithms usually need to model several sub-processes coexisting and communicating as a part of a distributed node, PlusCal processes must all be declared at top level and cannot be easily used to create what the algorithm is aiming for. 

\section{Distributed PlusCal}

\IEEEPARstart{D}{istributed PlusCal} is a language that extends PlusCal and is used to describe distributed algorithms.

\subsection{Structure of an algorithm}

The general structure and organization of a Distributed PlusCal algorithm is shown in Figure \ref{dpluscal-struct}.

\begin{figure}
\begin{lstlisting}[frame = tlrb, numbers = none, escapeinside={<@}{@>}]

(* <@\textcolor{brown}{algorithm}@> <algorithm name>

(* Declaration section *)
<@\textcolor{brown}{variables}@> <variable declarations>
<@\textcolor{blue}{channels}@> <channel declarations>
<@\textcolor{blue}{fifos}@> <fifo declarations>

(* Definition section *)
<@\textcolor{brown}{define}@> <definition name> == <definition description>

(* Macro section *)
<@\textcolor{brown}{macro}@> <name>(var1, ...)
 <macro-body of statements>

(* Procedure section *)
<@\textcolor{brown}{procedure}@> <name>(arg1, ...)
 variables <local variable declarations>
 <procedure body of statements>

(* Processes section *)
<@\textcolor{brown}{process}@> (<name> [=|\in] <Expr>))
  <@\textcolor{brown}{variables}@> <variable declarations>
 <@\textcolor{blue}{<sub-processes>}@>
*)
\end{lstlisting}

\caption{General structure of a Distributed PlusCal algorithm}
\label{dpluscal-struct}
\end{figure}

The \verb|PlusCal options section| holds options to be passed to the translator, we notify the translator to parse a Distributed PlusCal algorithm by passing the $-distpcal$ option.

\subsection{Communication Channels}
The \verb|declarations section| in Figure \ref{dpluscal-struct} allows the user to declare primitive constructs such as non-ordered channels and FIFO channels in addition to PlusCal variables.

The general structure for an unordered channel declaration is shown below. An unordered channel is declared with the keyword \keyword{channel} or \keyword{channels}.
\[
 \keyword{channel}\ \entity{id}[\entity{dimension1,...,dimensionN}];
\]

The corresponding \tlaplus translation of the unordered channel declared with N dimensions is shown below.
\[
 \entity{id} = [\entity{d1 \in dimension1,... dN \in dimensionN |-> \{ \}}];
\]

A FIFO channel is declared with the keyword \keyword{fifo} or \keyword{fifos}.  
\[
 \keyword{fifo}\ \entity{id}[\entity{dimension1,...,dimensionN}];
\]

The \tlaplus translation a FIFO channel is as follows
\[
 \entity{id} = [\entity{d1 \in dimension1,... dN \in dimensionN |-> \seq{}}];
\]

The $dimensions$ are \tlaplus sets that are defined in the \tlaplus file, they represent the keys of the channel. Defining a multidimensional channel corresponds to defining a set of channels.

The supported channel operators include \verb|send|, \verb|receive|, \verb|broadcast|, \verb|multicast|, \verb|clear|.

\subsection{Sub-Processes}
In the \verb|Process Section| in Figure \ref{dpluscal-struct} each process can hold multiple sub-processes each with its own body of statements. This enables the process to execute multiple tasks in parallel.

The listing below shows a small part of an example we developed for modeling the two phase commit protocol using Distributed PlusCal along with the \tlaplus translation produced by our translator.

 \begin{lstlisting}[frame = tlrb, numbers=none, escapeinside={<@}{@>}]  
    a3:<@\textcolor{brown}{await}@> (aState # <@\textcolor{blue}{"unknown"}@>);
       receive(agt[self], aState); 
\end{lstlisting}

\begin{lstlisting}[frame = tlrb, firstnumber=1, escapeinside={<@}{@>}]  
a3(self) == 
  /\ pc[self] [2] = <@\textcolor{blue}{"a3"}@>
  /\ (aState[self] # <@\textcolor{blue}{"unknown"}@>)
  /\ \E ag \in agt[self]:
        /\ aState' = [aState <@\textcolor{violet}{EXCEPT}@> ![self] = ag]
        /\ agt' = [agt <@\textcolor{violet}{EXCEPT}@> ![self] = agt[self] \ {ag}]
        /\ pc' = [pc <@\textcolor{violet}{EXCEPT}@> ![self] = [@  <@\textcolor{violet}{EXCEPT}@> ![2] =  <@\textcolor{blue}{"a4"}@>]]
        /\ <@\textcolor{violet}{UNCHANGED}@> << coord, cState, commits, msg >>
\end{lstlisting}

The value of the \verb|pc| variable in PlusCal is a single string equal to the label of the next statement to be executed with respect to a process. In Distributed PlusCal we extended the definition to indicate which sub-process is involved as well. In the listing above at line 2 is the \verb|pc| variable references the 2nd sub-process.

\section{Evalation}
In distributed algorithms several processes or threads may coexist and asynchronously communicate while being a part of a distributed node. Distributed PlusCal firstly introduced sub-processes to allow the declaration of these threads, and secondly we provided the user with communication channel constructs to assist in the communication between those processes.
\\\\
Extending the PlusCal translator allowed our translator to be backward compatible. The Distributed PlusCal translator can translate PlusCal models as well as Distributed PlusCal models.

\bibliographystyle{alpha}
\bibliography{report}
\end{document}
