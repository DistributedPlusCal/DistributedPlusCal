% Exemple d'utilisation de la classe `thesul' pour un master.
% D. Roegel, 3/4/2013
%
% Note : les couvertures de Master ne sont pas finalisées dans thesul.
%        Indiquez-moi ce qu'il faut mettre.
%
\documentclass{thesul}

\usepackage{capt-of}
\usepackage{placeins}
\usepackage[toc,page]{appendix}
\usepackage{multirow}
\usepackage{xspace}
\usepackage{latexsym}
\usepackage{amssymb}
\usepackage{listings}

\usepackage{color}
\usepackage[T1]{fontenc}

\definecolor{dkgreen}{rgb}{0,0.6,0}
\definecolor{gray}{rgb}{0.5,0.5,0.5}
\definecolor{mauve}{rgb}{0.58,0,0.82}
\definecolor{orange}{rgb}{1.0,0.49,0.0}

\usepackage{enumitem}
\usepackage{pifont}

\lstset{frame=single,
  language={Java},
  aboveskip=3mm,
  belowskip=3mm,
  showstringspaces=false,
  basicstyle={\small\ttfamily},
  numbers=left,
  stepnumber=1,
  numberstyle=\tiny\color{gray},
  keywordstyle=\color{orange},
  commentstyle=\color{dkgreen},
  breaklines=true,
  breakatwhitespace=true,
  tabsize=1,
  morecomment=[l]{\\*},
  morekeywords={fair, process, macro, procedure, define, algorithm, variables}
}
\newcommand\tab[1][1cm]{\hspace*{#1}}
\renewcommand{\floatpagefraction}{.99}
\renewcommand{\textfraction}{.01}

\newcommand{\tlaplus}{TLA\textsuperscript{+}\xspace}
\newcommand{\EXCEPT}{\textsc{except}}

%% make TeX use text italic font in math environments
\makeatletter
\newcounter{abr@ctr}
\newcommand{\abr@c}{\c@abr@ctr\advance\c@abr@ctr\@ne}

\DeclareSymbolFont{tlaitalics}{\encodingdefault}{cmr}{m}{it}
\let\itfam\symtlaitalics

\newcommand{\noTeXmath}{%
\c@abr@ctr=\itfam
\multiply\c@abr@ctr"100\relax
\advance\c@abr@ctr "7061\relax
\mathcode`a=\abr@c\mathcode`b=\abr@c\mathcode`c=\abr@c\mathcode`d=\abr@c
\mathcode`e=\abr@c\mathcode`f=\abr@c\mathcode`g=\abr@c\mathcode`h=\abr@c
\mathcode`i=\abr@c\mathcode`j=\abr@c\mathcode`k=\abr@c\mathcode`l=\abr@c
\mathcode`m=\abr@c\mathcode`n=\abr@c\mathcode`o=\abr@c\mathcode`p=\abr@c
\mathcode`q=\abr@c\mathcode`r=\abr@c\mathcode`s=\abr@c\mathcode`t=\abr@c
\mathcode`u=\abr@c\mathcode`v=\abr@c\mathcode`w=\abr@c\mathcode`x=\abr@c
\mathcode`y=\abr@c\mathcode`z=\abr@c
\c@abr@ctr=\itfam
\multiply\c@abr@ctr"100\relax
\advance\c@abr@ctr "7041\relax
\mathcode`A=\abr@c\mathcode`B=\abr@c\mathcode`C=\abr@c\mathcode`D=\abr@c
\mathcode`E=\abr@c\mathcode`F=\abr@c\mathcode`G=\abr@c\mathcode`H=\abr@c
\mathcode`I=\abr@c\mathcode`J=\abr@c\mathcode`K=\abr@c\mathcode`L=\abr@c
\mathcode`M=\abr@c\mathcode`N=\abr@c\mathcode`O=\abr@c\mathcode`P=\abr@c
\mathcode`Q=\abr@c\mathcode`R=\abr@c\mathcode`S=\abr@c\mathcode`T=\abr@c
\mathcode`U=\abr@c\mathcode`V=\abr@c\mathcode`W=\abr@c\mathcode`X=\abr@c
\mathcode`Y=\abr@c\mathcode`Z=\abr@c}

\newcommand{\TeXmath}{%
\mathcode`a="7161\mathcode`b="7162\mathcode`c="7163\mathcode`d="7164%
\mathcode`e="7165\mathcode`f="7166\mathcode`g="7167\mathcode`h="7168%
\mathcode`i="7169\mathcode`j="716A\mathcode`k="716B\mathcode`l="716C%
\mathcode`m="716D\mathcode`n="716E\mathcode`o="716F\mathcode`p="7170%
\mathcode`q="7171\mathcode`r="7172\mathcode`s="7173\mathcode`t="7174%
\mathcode`u="7175\mathcode`v="7176\mathcode`w="7177\mathcode`x="7178%
\mathcode`y="7179\mathcode`z="717A\mathcode`A="7141\mathcode`B="7142%
\mathcode`C="7143\mathcode`D="7144\mathcode`E="7145\mathcode`F="7146%
\mathcode`G="7147\mathcode`H="7148\mathcode`I="7149\mathcode`J="714A%
\mathcode`K="714B\mathcode`L="714C\mathcode`M="714D\mathcode`N="714E%
\mathcode`O="714F\mathcode`P="7150\mathcode`Q="7151\mathcode`R="7152%
\mathcode`S="7153\mathcode`T="7154\mathcode`U="7155\mathcode`V="7156%
\mathcode`W="7157\mathcode`X="7158\mathcode`Y="7159\mathcode`Z="715A}

\noTeXmath
\makeatother

\makeatletter
\newcommand*{\centerfloat}{%
  \parindent \z@
  \leftskip \z@ \@plus 1fil \@minus \textwidth
  \rightskip\leftskip
  \parfillskip \z@skip}
\makeatother

% -------------------------------------------------------------------
%                         Les références
%-------------------------------------------------------------------

\NoChapterNumberInRef
\NoChapterPrefix


\begin{document}

      \OddHead={{\leftmark\rightmark}{\hfil\slshape\rightmark}}
      \EvenHead={{\leftmark}{{\slshape\leftmark}\hfil}}
      \OddFoot={\hfil\thepage}
      \EvenFoot={\thepage\hfil}
      \pagestyle{ThesisHeadingsII}

%-------------------------------------------------------------------
%            Réinitialisation de la numérotation des chapitres
%-------------------------------------------------------------------

% Si la commande suivante est présente,
% elle doit figurer APRÈS \begin{document}
% et avant la première commande \part
\ResetChaptersAtParts 




\MasterUL
\ThesisTitle{Formal Verification of Distributed Algorithms using Distributed-PlusCal}
\ThesisAuthor{Heba Al-kayed}
\ThesisPresentedThe{soutenu le 26 juin 2020}
\President    = {Le pr\'esident}
\Rapporteurs  = {Le rapporteur 1\\
                 Le rapporteur 2\\
                 Le rapporteur 3}
\Examinateurs = {L'examinateur 1\\
                 L'examinateur 2}

\MakeThesisTitlePage


%-------------------------------------------------------------------
%                          remerciements
%-------------------------------------------------------------------

%\DontFrameThisInToc
\begin{ThesisAcknowledgments}
Les remerciements.
\end{ThesisAcknowledgments}

%-------------------------------------------------------------------
%                            dédicace
%-------------------------------------------------------------------

\begin{ThesisDedication}
Je dédie cette thèse\\
à ma machine.\\
Oui, à Pandore,\\
qui fut la première de toutes.
\end{ThesisDedication}


%-------------------------------------------------------------------
%                  écriture de `Chapitre' et `Partie' 
%                      dans la table des matières
%-------------------------------------------------------------------

\WritePartLabelInToc
\WriteChapterLabelInToc

%-------------------------------------------------------------------
%                        table des matières
%-------------------------------------------------------------------

\tableofcontents

% Pour ne pas avoir le mot « Chapitre » au début de chaque chapitre.
\NoChapterHead


% La commande \mainmatter permet de passer
% à la numérotation arabe (ce que fait \pagenumbering{arabic}) 
% et de faire commencer la nouvelle page 1 sur une page impaire.
% On évitera donc d'utiliser directement \pagenumbering{arabic}.
\mainmatter


\chapter{Introduction}
%%============================

- Model checking is a verification method that is automatic and model-based(system is represented by a model, a specification is represented as a formula and we check whether the model satisfies the formula ==> if not we have counter models), it is intended to be used for concurrent systems.

- The purpose of a modeling language is to describe what a system must perform,
not how a system must perform.
- the users of these modeling languages are algorithm designers who
are responsible for describing the functionality of the system in terms of algorithms
before actual implementation. Thus, these languages should be simple so that the
users can learn and use the language constructs easily.

- PlusCal language by Leslie Lamport provides simple pseudo-code like interface
for the user to express concurrent systems.

- distributed system bugs are difficult to find by testing, they tend to be non-reproducible or not covered by test-cases.

- TLA+
- 

\chapter{Background info}
%%============================

This chapter is an overview of TLA+ and PlusCal.
\section{\tlaplus}
%%============================

\tlaplus is a formal specification language in which algorithms and systems can be described at a high level of abstraction and can be formally verified using the model checker TLC or the interactive proof assistant TLAPS. \tlaplus is based on mathematical set theory for describing data structures in terms of sets and functions, and on the Temporal Logic of Actions TLA for specifying their executions as state machines. \tlaplus specifications usually have the form
\[
  Init \land \Box[Next]_{vars} \land L
\]
where $Init$ is a predicate describing the possible initial states, $Next$ is a predicate that constrains the possible state transitions, $vars$ is the tuple of all state variables that appear in the specification, and $L$ is a liveness or fairness property expressed as a formula of temporal logic. Transition formulas such as $Next$, also called \emph{actions}, are at the core of \tlaplus, and represent instantaneous state changes. They contain unprimed state variables denoting the value of the variable before the transition as well as primed state variables that denote the value after the transition.

\begin{figure}
\begin{lstlisting}% [caption = A memory specification in \tlaplus., frame = tlrb, firstnumber =  1,label=memory-tla]

---------------------------- MODULE SimpleMemory ----------------------------
CONSTANTS Address, Value, InitValue, NoValue

ASSUME 
  /\ InitValue \in Value
  /\ NoValue \notin Value

VARIABLES chan, mem

\* initial condition
Init == 
  /\ chan = NoValue
  /\ mem = [a \in Address |-> InitValue]

\* transitions: reading and writing
Read(a) == 
  /\ chan' = mem[a]
  /\ mem' = mem

Write(a,v) ==
  /\ mem' = [mem EXCEPT ![a] = v]
  /\ chan' = NoValue

Next ==
  \/ \E a \in Address : Read(a)
  \/ \E a \in Address, v \in Value : Write(a,v)

\* overall specification
Spec == Init /\ [][Next]_<<chan,mem>>

\* predicate specifying type correctness
TypeOK == 
  /\ chan \in Value \cup {NoValue}
  /\ mem \in [Address -> Value] 
=============================================================================
\end{lstlisting}
\caption{A memory specification in \tlaplus.}
\label{memory-tla}
\end{figure}

For example, figure~\ref{memory-tla} shows a \tlaplus specification of a simple memory. It declares four constant parameters $Address$, $Value$, $InitValue$, and $NoValue$, and states a hypothesis on the values that these parameters can be instantiated with. The state space of the specification is represented by the two variables $chan$ and $mem$. Intuitively, $mem$ holds the current memory, whereas $chan$ is an output channel that reflects the result of the preceding operation.

The remainder of the \tlaplus module contains operator definitions that represent parts of the specification and of correctness properties. The state predicate $Init$ fixes the initial values of the two variables. The actions $Read(a)$ and $Write(a,v)$ represent reading the value at memory address $a$ and writing value $v$ to memory address $a$, respectively. In this specification, the memory is modeled as a function mapping addresses to values. The \tlaplus expression $[x \in S \mapsto e]$ denotes the function with domain $S$ such that every element $x$ of $S$ is mapped to $e$. This is reminiscent of a $\lambda$-expression but also makes explicit the domain of the function. Function application $f[x]$ is written using square brackets. Finally, the expression $[f\ \EXCEPT\ ![x] = e]$ denotes the function that is similar to $f$, except that argument $x$ is mapped to $e$, one can think of it as a function overwrite.

The action $Next$ defines the possible state transitions as the disjunction of $Read$ and $Write$ actions, and $Spec$ represents the overall specification of the memory.

\tlaplus is an untyped language. Type correctness can be verified as a property of the specification. For our example, the predicate $TypeOK$ indicates the possible values that the variables $chan$ and $mem$ are expected to hold at any state of the specification. Formally, the implication $Spec \Rightarrow \Box TypeOK$ can be established as a theorem.

\bigskip


For More details on the syntax and grammer of TLA+, see \ref{tlabook}.

\section{PlusCal algorithm language}
%%============================

\tlaplus is a specification formalism not a programming language. It relies on mathematical logic and formulas for structuring specifications, which may discourage programmers from using it. PlusCal~\cite{pcalAlgo} was designed as an algorithm language with a more familiar syntax that can be translated into \tlaplus specifications and then be verified using the familiar \tlaplus tools.

An algorithm language is used to focus on aspects of the algorithm such as data manipulation rather than irrelevant and distracting details that involve programming-language objects and data structures.
	
PlusCal is an algorithm language that describes both concurrent and sequential algorithms, it maintains the expressiveness of \tlaplus as well as representing atomicity conveniently.

The TLA+ Toolbox provides a platform where algorithm designers can model their algorithms using PlusCal, translate them to the corresponding \tlaplus specifications and check for the algorithm's correctness through the TLC model checker.

A PlusCal algorithm is located in a comment statement within the \tlaplus module. The general structure of a PlusCal algorithm is shown in Figure \ref{pluscal-struct}.

\begin{figure}[!h]
\begin{lstlisting}% [caption = General structure of a PlusCal algorithm, frame = tlrb, firstnumber =  1,label=pluscal-struct]

(** algorithm <algorithm name>

(* Declaration section *)
variables <variable declarations>

(* Definition section *)
define <definition name> == <definition description>

(* Macro section *)
macro <name>(var1, ...)
 <macro body of statements>

(* Procedure section *)
procedure <name>(arg1, ...)
 variables <local variable declarations>
 <procedure body of statements>

(* Processes section *)
process (<name> [=|\in] <expr>))
  variables <variable declarations>
  <process body of statements>

**)

\end{lstlisting}
\caption{General structure of a PlusCal algorithm}
\label{pluscal-struct}
\end{figure}

The \textbf{Declaration section} is where the user declares global variables that are shared among all the components of the algorithm. The \textbf{Definition section} allows the user to write TLA+ definitions of operators that may refer to the algorithm's global variables.
The \textbf{Macro section} holds macros whose bodies are expanded at translation time incorporating the parameters passed from the calling statement, similar to the expansion of C pre-processing macros. A \textbf{procedure} in PlusCal take a number of arguments, can declare local variables, and can modify the global variables; it does not return a result. A \textbf{process} begins in one of two ways: 
\begin{center}
$process (ProcName \in IdSet)$ \\
$process (ProcName = Id)$
\end{center}

The first form declares a set of processes, the second an individual process. these statements are optionally followed by declarations of local variables. The process body is a sequence of statements, within the body of a process set, $self$ equals the current process's identifier.

Procedure and process bodies may contain labels. All PlusCal statements appearing between two labels are executed atomically, and certain rules determine where labels must and may not appear.
PlusCal enforces a strict ordering of its blocks. The define block has to come before any macros, which has to come before any procedures, which has to come before any processes. The full grammar of the PlusCal algorithm language can be found in appendix A of the PlusCal manual~\cite{pcalAlgo}.

Figure \ref{semaphoreEx} shows the modeling of a semaphore mutex example in PlusCal, Semaphores are integer variables that are used to solve the critical section problem.

\FloatBarrier 
\begin{figure}
\begin{lstlisting}[frame = tlrb, firstnumber = 1]
(*
--algorithm SemaphoreMutex {
variables sem = 1;

  process(p \in 1..N)
  {
  start : while (TRUE){
  enter :    when (sem > 0);
                  sem := sem - 1;            
  cs :       skip ;
  exit :     sem := sem + 1 ;
          }
  }
}
*)

\end{lstlisting}
\caption{Semaphore mutex example in PlusCal}
\label{semaphoreEx}
\end{figure} 

\paragraph{Translation to \tlaplus.}
The PlusCal translator expects as input a PlusCal algorithm following the structure described previously.

% The translator
It parses the PlusCal algorithm and generates the corresponding \tlaplus specification in roughly the following steps.

\begin{enumerate}
\item Generate all the definitions and variables regardless of their scope within the PlusCal algorithm, as well as $vars$ the tuple of all variables.
The $pc$ variable is introduced by the translator to track the control flow of processes.

\FloatBarrier
\begin{lstlisting}[frame = tlrb, firstnumber = 1,xleftmargin=.2\textwidth, xrightmargin=.2\textwidth]
\* BEGIN TRANSLATION
VARIABLES sem, pc

vars == << sem, pc >>

\end{lstlisting}

\item Generate $ProcSet$ which is a set that contains all the process identifiers.

\begin{lstlisting}[frame = tlrb, firstnumber = 1,xleftmargin=.2\textwidth, xrightmargin=.2\textwidth]
ProcSet == (1..N)

\end{lstlisting}

\item  Generate $Init$, the initial predicate that specifies the initial values of all the declared variables. Comments indicate if the variables are global or local to a process or procedure.
The variable $pc$ is defined and used as a program control variable, it's a function whose domain is $ProcSet$ such that each element is mapped to the entry label of the process.

\begin{lstlisting}[frame = tlrb, firstnumber = 1, xleftmargin=.2\textwidth, xrightmargin=.2\textwidth]
Init == (* Global variables *)
        /\ sem = 1
        /\ pc = [self \in ProcSet |-> "st"]

\end{lstlisting}

\item For each PlusCal label, generate a \tlaplus action that represents the atomic operation beginning at that label. 
In the produced actions unprimed variables refer to their values before executing the action and the primed variables refer to their values after the execution.
The definition is parameterized by the identifier $self$,
which represents the identifier for the current process. For example, the following action is generated for label \texttt{start} of the semaphore algorithm.

\begin{lstlisting}[frame = tlrb, firstnumber = 1, xleftmargin=.2\textwidth, xrightmargin=.2\textwidth]
enter(self) == /\ pc[self] = "enter"
               /\ (sem > 0)
               /\ sem' = sem - 1
               /\ pc' = [pc EXCEPT ![self] = "cs"]
\end{lstlisting}

Moreover, the PlusCal translator generates an action that corresponds to the disjunction of the actions for the individual labels and that represents the transition relation of a process.

\begin{lstlisting}[frame = tlrb, firstnumber = 1, xleftmargin=.05\textwidth, xrightmargin=.05\textwidth]
p(self) == start(self) \/ enter(self) \/ cs(self) \/ exit(self)
\end{lstlisting}

\item Generate the next-state action $Next$ that constrains the possible state transitions, and the complete specification $Spec$.

\begin{lstlisting}[frame = tlrb, firstnumber = 1, xleftmargin=.2\textwidth, xrightmargin=.2\textwidth]
Next == (\E self \in 1..N: p(self))

Spec == Init /\ [][Next]_vars

\end{lstlisting}

\end{enumerate}


In practice, to use PlusCal the user must understand the generated TLA+ specifications, in order to write the properties in terms of the TLA+ variables introduced by the compiler.

A more detailed description of the translation strategy can be found in \cite{pcalAlgo}.

\chapter{Related work}
%%============================

- pluscal extensions for generating code to make sure that verified spec isn't lost while being implemented
- \ref{PGO} a tool represented as a part of a master thesis, it aims to produce an implementation in Go(C based language developed by Google) based on a PlusCal/TLA+ specification.

\section{PGO}

PGo is a source to source compiler written in Java. It compiles specifications written in an extension of PlusCal, called Modular PlusCal to Go programs, PGo can compile Modular PlusCal to PlusCal, PlusCal to Go, and Modular PlusCal to Go.

\subsection{Modular PlusCal}

Modular PlusCal is an extension of PlusCal, using Modular PlusCal the user can separate the specification into two components, a system functionality specification concerned with what the algorithm is supposed to achieve and an environment specification concerned about how we may want it achieved.

\hfill\\

For example, if we consider a server/client based communication system, the system functionality can be a client requesting services, this is called system functionality, which is not related to how the client is requesting that service via a TCP connection for example, this is called an environment specification. Another added value for this separation is that the programmer can reuse concept defined in one specification in another specification since the environment specification isn't dependent on the functionality of the system.

Modular PlusCal introduced Architypes, Mapping Macros, and Instances to achieve the modularization of the spec.

\paragraph{\textbf{Architypes}}
\hfill\\
They are considered to be the blue print of a PlusCal process, they are used to specify the system behaviors. They have the same semantics except for the inability to access global variables unless passed to the architypes as arguments.

Using the \textit{ref} keyword before a global variable that is sent as an argument lets us know that this architype can modify it.

these restrictions provide the needed isolation between system and environment behaviours.

\paragraph{Mapping Macros}

They specify the environment, they define interfaces for reading and writing on a global variable that represents a network.
like pluscal macros they cannot contain any labels inside them, all statements are apart of the same step.

Modeling mapping macros independently from system functionality allows the user to reuse the mapping macros.

\paragraph{Instances}

Instances are the glue that holds the archtypes and mapping macros together.
instantiates a process or a group of processes using the specified architype, each argument passed is bound with a mapping macro to control read and write functions on it.

- show a smaller example or grammer

\chapter{The Distributed-PlusCal language}


Distributed algorithms are based on continuous interactions among components, they benefit greatly from testing failure conditions like deadlocks or race conditions at early stages of development at design level.
TLA+ provides a flexibility and an expressiveness that enables it specify and verify those algorithms. One of the popular modern examples of incorporating TLA+ to verify distributed algorithms is its usage at Amazon Web Services \cite{amazon}.

Distributed-PlusCal is a language used to describe distributed algorithms, it extends PlusCal. 
Our motivations for creating Disrtibuted PlusCal are quite similar to the motivations that created PlusCal, we want a syntax that would spare the user from having to model primitives that usually accompany distributed algorithms.

Since we extended the existing PluCal translator responsible for parsing PlusCal into TLA+ we inherited the same semantics and grammar and added our own.

\section{Structure of an algorithm}

The general structure and organization of a Distributed-PlusCal algorithm is shown in Figure \ref{dpluscal-struct}.


\begin{figure}
\begin{lstlisting}% [caption = General structure of a Distributed-PlusCal algorithm, frame = tlrb, firstnumber =  1,label=dpluscal-struct]

(* PlusCal options section *)
(* PlusCal options (-distpcal) *)

(* algorithm <algorithm name>

(* Declaration section *)
variables | channels | FIFOs <variable declarations>

(* Definition section *)
define <definition name> == <definition description>

(* Macro section *)
macro <name>(var1, ...)
 <macro-body of statements>

(* Procedure section *)
procedure <name>(arg1, ...)
 variables | channels | FIFOs <local variable declarations>
 <procedure body of statements>

(* Processes section *)
process (<name> [=|\in] <Expr>))
  variables | channels | FIFOs <variable declarations>
  <sub-processes>

*)

\end{lstlisting}
\caption{General structure of a Distributed-PlusCal algorithm}
\label{dpluscal-struct}
\end{figure}

\textbf{PlusCal options section} holds options to be passed to the translator, for example adding $-label$ to the PlusCal options turns on the automatic labeling of the algorithm by the translator. Since we extended the PlusCal translator we notify the translator to parse a Distributed-PlusCal algorithm by passing the $-distpcal$ option.

Figure \ref{dpluscal-struct} resembles figure \ref{pluscal-struct}, however the variable declarations in \textbf{declarations, procedures, and processes} sections enables the user to declare primitive constructs such as non-ordered channels and FIFO based channels in addition to PlusCal variables.More details on the communication channels are available in section \ref{comChannels}.

In the \textbf{Process Section} the process can hold multiple sub-processes each with it's own body of statements.More details on the sub-processes are available in section \ref{subProcess}.


\hfill\\
In the sections that follow we will be doing a walk-through on the two phase commit protocol \ref{appendix:examples}, it is used in distributed transactions that consist of multiple operations, performed at multiple sites. The goal of the protocol is to reach consensus between the different elements that carry out the transaction together such that if an element decides to abort all other elements abort as well, and the transaction is rolled back.

\FloatBarrier
\section{Communication Channels}
\label{comChannels}

Distributed algorithms composed of multiple processes communicate by exchanging messages through channels. Channels provide communication and coordination for distributed applications. They are classified by the way they handle the addition and removal of messages from their collections. 

For some algorithms message ordering is critical, for example if a message is sent to subscribe in a service and another is send to unsubscribe, it is crucial that the first message is received and handled before the second. However, some algorithms are not dependent on ordering. Distributed-PlusCal offers two types of channels, unordered channels as well as FIFO based channels.

Channels in Distributed-PlusCal are unbounded , meaning there is no maximum capacity for the messages they can hold.

The sections to follow explain how to define and use the channels as well as their \tlaplus translations. 

\subsection{Unordered Channels}

Unordered channel are represented by \tlaplus sets.\tlaplus sets are the basic construct for representing data. Sets are not concerned with the order of elements within their collection.

\tlaplus sets can handle complex structures for example you can have a set of strings, integers or even a set of sets. \{ \} represents an empty set. 

Set operators include $\in$ which is written as \lstinline|\in| in \tlaplus checks whether an element is a part of a set, $\cup$ is written as \lstinline|\cup| adds an element to the set.

The general structure for channel declaration is shown below. An unordered channel is declared with the keyword $channel$ or $channels$. 

\[
 channel <identifier>[<dimentions>];
\]


The dimensions are \tlaplus sets that are defined in the \tlaplus file, they represent the keys of the channel, Figure \ref{unorderedtla} shows the corresponding translation of the unordered channel deceleration with N dimensions.


\FloatBarrier
\begin{figure}[!h]
\begin{lstlisting}% [caption = General structure of a Distributed-PlusCal algorithm, frame = tlrb, firstnumber =  1,label=dpluscal-struct]

<channel-identifier> = [s1 \in IdSet1, s2 \in IdSet2, ..., sN \in IdSetN |-> {}]

\end{lstlisting}
\caption{Unordered channel declaration - TLA+ translation}
\label{unorderedtla}
\end{figure}
\FloatBarrier


In our two phase commit example, we have two unordered channels $coord$ and $agt$, their Distributed-PlusCal declarations and \tlaplus translation are in Figure \ref{2pcchannels}.

\FloatBarrier
\begin{figure}[!h]
\begin{lstlisting}% [caption = General structure of a Distributed-PlusCal algorithm, frame = tlrb, firstnumber =  1,label=dpluscal-struct]

(***
----------------------------- MODULE 2pc -----------------------------
EXTENDS Sequences, Naturals
CONSTANTS Coord, Agent
(* PlusCal options (-distpcal) *)
  
--algorithm TPC { 

  \* message channels
  channels coord, agt[Agent];
  .
  .
  .
 }
*)
  
  \* BEGIN TRANSLATION 
  .
  .
  .
  Init == (* Global variables *)
        /\ coord = {}
        /\ agt = [a0 \in Agent |-> {}]
  .
  .
  .      
  \* END TRANSLATION 
***)

\end{lstlisting}
\caption{Unordered channel declaration - TLA+ translation}
\label{2pcchannels}
\end{figure}
\FloatBarrier

Distributed-PlusCal extended PlusCals allowed statements within a body to include channel operators.
The list below shows the syntax of the channel operators and their \tlaplus translations, to simplify the examples assume we are dealing with a one dimensional channel with the identifier $chan$.

\begin{itemize}
\item[$\Diamond$]  \textbf{send} The $send$ operator carries two parameters, the channel $chan$ and the message $msg$. The purpose of this operator is to add the message $msg$ to
the given channel. Since the unordered channel is represented using a \tlaplus set, addition to a set is performed using the set operator $\cup$ which is written in \tlaplus as \lstinline|\cup|.


\begin{minipage}{.3\textwidth}

\lstinline|send(chan[element], msg)| $\triangleq$\\
\end{minipage}\hfill
\begin{minipage}{.7\textwidth}
\begin{lstlisting}[frame = none, numbers = none]
chan' = [chan EXCEPT ![element] = chan[element] \cup {msg}]
\end{lstlisting}

\end{minipage}\hfill
where $element \in IdSet$.
\item[$\Diamond$]  \textbf{broadcast}

The $broadcast$ operator handles two parameters, the channel $chan$ and the message $msg$. The purpose of this operator is to add the message $msg$ to every element of the given channel.

\begin{minipage}{.25\textwidth}

\lstinline|broadcast(chan, msg)| $\triangleq$
\end{minipage}\hfill
\begin{minipage}{.7\textwidth}
\begin{lstlisting}[frame = none, numbers = none]
chan' = [el \in IdSet ->chan[el] \cup {msg}]
\end{lstlisting}

\end{minipage}\hfill

Note that we don't specify an element when referring to the channel in the first argunemt, as the message is meant to be sent to all element.
\item[$\Diamond$] \textbf{multicast}

The $multicast$ operator handles two parameters, the channel $chan$ and a \tlaplus expression. The second parameter specifies the intended recipients in the channel and the message. The purpose of this operator is to identify the recipients such that if they fall within the required subset, the message is added to their sets; otherwise, their sets remain unchanged.

\begin{minipage}{.44\textwidth}

\lstinline|multicast(chan, [a \in sub -> msg])| $\triangleq$\\\\\\
\end{minipage}\hfill
\begin{minipage}{.6\textwidth}
\begin{lstlisting}[frame = none, numbers = none]
chan'=[el \in DOMAIN chan 
			|-> IF el \in sub 
       THEN chan[el] \cup {msg} 
       ELSE chan[el]
\end{lstlisting}

\end{minipage}\hfill

\item[$\Diamond$]  \textbf{receive}
The $receive$ operator has two parameters, the channel $chan$ and a \tlaplus variable $var$. The purpose of this operator is to remove an element from the channel and add it to the variable sent in the second argument.

\begin{minipage}{.35\textwidth}

\lstinline|receive(chan[element], var)| $\triangleq$\\\\\\
\end{minipage}\hfill
\begin{minipage}{.6\textwidth}
\begin{lstlisting}[frame = none, numbers = none]
chan' = \E el \in chan[element]:
        /\ var' = el
        /\ chan' = [chan EXCEPT ![element] = chan[element] \ {el}]
\end{lstlisting}

\end{minipage}\hfill

\item[$\Diamond$]  \textbf{clear}
The $clear$ operator empties the channel $chan$.

\lstinline|clear(chan)| $\triangleq$
\lstinline|chan' = [el \in IdSet -> {}]|

\end{itemize}

The two phase commit example in \ref{appendix:examples} includes the functions $send$, $broadcast$, $receive$ and $clear$ with their \tlaplus translations.

%example for multicast%
\subsection{FIFO channels}

\tlaplus sequences contain operators that enables iterating over all their elements and keeping track of the order between the elements. For instance, adding to a sequence corresponds to appending at the end of the sequence with the $Append$ operator, and removing the first element of a sequence can be accomplished with the $Tail$ operation. FIFO channels are represented as \tlaplus sequences. We take advantage of these operators to maintain the order in the FIFO channels.

All sequences are \tlaplus functions, where $<<>>$ represents an empty sequence.

The general structure for channel declaration is shown below. A FIFO channel is declared with the keyword $fifo$ or $fifos$. 

\begin{minipage}{.35\textwidth}

\lstinline|fifo <identifier>[<dimentions>]| $\triangleq$
\end{minipage}\hfill
\begin{minipage}{.65\textwidth}
\begin{lstlisting}[frame = none, numbers = none]

<identifier> = [s1 \in IdSet1, s2 \in IdSet2,...,
					 sN \in IdSetN |-> <<>>]
\end{lstlisting}

\end{minipage}\hfill

%dimensions already explained%

Appendix \ref{appendix:examples} includes an example for  modeling Lamport's Mutual Exclusion algorithm, an algorithm that uses time-stamp to order critical section requests and to resolve any conflict between requests. the example introduces a FIFO channel $network$ and performs the supported FIFO channel operations. \\

The Distributed-PlusCal declaration and the \tlaplus translation of the $network$ FIFO channel are in Figure \ref{fifochannels}.

\FloatBarrier
\begin{figure}[!h]
\begin{lstlisting}% [frame = tlrb, firstnumber =  1]

(***
----------------------------- MODULE LamportMutex -----------------------------
EXTENDS Sequences, Naturals, TLC
CONSTANT N
ASSUME N \in Nat 
Nodes == 1 .. N

(* PlusCal options (-distpcal) *)
  
--algorithm LamportMutex { 

  fifo network[Nodes, Nodes];
  .
  .
  .
 }
*)
  
  \* BEGIN TRANSLATION 
  .
  .
  .
  Init == (* Global variables *)
        /\ network = [n0 \in Nodes, n1 \in Nodes |-> <<>>]
  .
  .
  .      
  \* END TRANSLATION 
***)

\end{lstlisting}
\caption{FIFO channel declaration - TLA+ translation}
\label{fifochannels}
\end{figure}
\FloatBarrier


Distributed-PlusCal extended PlusCals allowed statements within a body to include functions that are applied on channels.
The list below shows the syntax of the channel operators and their \tlaplus translations, to simplify the examples assume we are dealing with a one dimensional channel with the identifier $chan$.

\begin{itemize}

\item[$\Diamond$]  \textbf{send} The $send$ operator carries two parameters, the channel $chan$ and the message $msg$. The purpose of this operator is to add the message $msg$ to
the given channel. Since the FIFO channel is represented using a \tlaplus sequence, addition to a sequence is performed using the operator $Append$ which adds to the tail of the sequence.

\begin{minipage}{.3\textwidth}

\lstinline|send(chan[element], msg)| $\triangleq$
\end{minipage}\hfill
\begin{minipage}{.7\textwidth}
\begin{lstlisting}[frame = none, numbers = none]
chan' = [chan EXCEPT ![element] = Append(@, msg)]\end{lstlisting}

\end{minipage}\hfill
where $element \in IdSet$.


\item[$\Diamond$]  \textbf{broadcast}
The $broadcast$ operator handles two parameters, the channel $chan$ and the message $msg$. The purpose of this operator is to add the message $msg$ to every element of the given channel.

\begin{minipage}{.25\textwidth}

\lstinline|broadcast(chan, msg) | $\triangleq$
\end{minipage}\hfill
\begin{minipage}{.7\textwidth}
\begin{lstlisting}[frame = none, numbers = none]
chan' = [el \in IdSet -> Append(chan[el], msg)]
\end{lstlisting}

\end{minipage}\hfill

\item[$\Diamond$] \textbf{multicast}

The $multicast$ operator handles two parameters, the channel $chan$ and a \tlaplus expression. The second parameter specifies the intended recipients in the channel and the message. The purpose of this operator is to identify the recipients such that if they fall within the required subset, the message is added to their sets; otherwise, their sets remain unchanged.

\begin{minipage}{.44\textwidth}

\lstinline|multicast(chan, [a \in sub -> msg])| $\triangleq$\\\\\\
\end{minipage}\hfill
\begin{minipage}{.6\textwidth}
\begin{lstlisting}[frame = none, numbers = none]
chan'=[el \in DOMAIN chan 
			|-> IF el \in sub 
       THEN Append(chan[el], msg)
       ELSE chan[el]
\end{lstlisting}

\end{minipage}\hfill


\item[$\Diamond$]  \textbf{receive}

The $receive$ operator has two parameters, the channel $chan$ and a \tlaplus variable $var$. The purpose of this operator is to remove the first element from the channel using the $Head$ and $Tail$ operators and add it to the variable sent in the second argument.

\begin{minipage}{.3\textwidth}

\lstinline|receive(chan[element],var)| $\triangleq$\\\\
\end{minipage}\hfill
\begin{minipage}{.7\textwidth}
\begin{lstlisting}[frame = none, numbers = none]
chan' = Len(chan[element]) > 0 
        /\ var' = [Head(chan[element])]
        /\ chan' = [chan EXCEPT ![element] =  Tail(@) ]
\end{lstlisting}

\end{minipage}\hfill

\item[$\Diamond$]  \textbf{clear}
The $clear$ operator empties the channel $chan$.

\begin{center}

\lstinline|clear(chan)| $\triangleq$
\lstinline|chan' = [el \in IdSet -> <<>>]|

\end{center}
\end{itemize}

The Lamport Mutex example in \ref{appendix:examples} includes the operators $send$, $multicast$, $receive$ and $clear$ with their \tlaplus translations.

\section{Sub-Processes}
\label{subProcess}

Distributed-PlusCal gives the user the opportunity to define multiple sub-processes per process. Each sub-process consists of labeled statements. Essentially this enables the process execute multiple tasks in parallel.

The body of a sub-process maintains the same syntax as the body of a PlusCal process. All the sub-processes share the same variables declared for the process, this makes communication between them possible if needed.

The Figure \ref{2pcSub} shows a sub-process defined for the two phase commit algorithm, the example is written in c-syntax and the sub-processes are surrounded by curly braces.
The entire algorithm is in Appendix \ref{appendix:examples}.

\begin{figure}[!h]
\label{2pcSub}
\begin{lstlisting}[caption = Distributed-PlusCal Sub-Processes, frame = tlrb, firstnumber = 1]
   
  fair process (a \in Agent)
  variable aState = "unknown"; { \* sub-process

a1: if (aState = "unknown") {
        with(st \in {"accept", "refuse"}) {
          aState := st;
          send(coord, [type |-> st, agent |-> self]);
        };
    };
    a2: await(aState \in {"commit", "abort"})
    
  } { \* sub-process
    
    a3:await (aState # "unknown");
       receive(agt[self], aState); 
       
    a4:clear(agt);
  }

\end{lstlisting}
\end{figure}

\hfill\\

The process represents an Agent process that communicates with a coordinator in order to apply a decision. The process consists of two sub-processes, where the first sub-process contains the atomic labels <<a1, a2>> and the second sub-process contains <<a3, a4>>.

it's important to note that the sub-processes do not allow variable declarations they only use variables declared for the entire process.

\subsection{TLA+ Translation}

The translation of a Distributed-PlusCal process into TLA+ meant having to introduce structures and sets that refer to the sub-processes as well as the processes.

Initially a set \textbf{SubProcSet} was added to hold all the sub-process identifiers, with respect to which process they belonged to.

\[
SubProcSet == [P \in ProcSet |-> IF P \in IdSet  THEN  1..n
								  ELSE  1..m ]
\]						  

The value of the \textbf{pc} variable in PlusCal is a single string equal to the label of the next statement to be executed with respect to a process. In Distributed-PlusCal we extended the definition to indicate which sub-process is involved as well.

Since in \tlaplus a function with domain 1..n for some n in Nat is a sequence, the \textbf{pc} variable in Distributed-PlusCal is initialized as :

\[
pc = [self \in ProcSet |-> self \in IdSet -> <<"action",...>>]
\]	

where actions in the sequence are the entry point actions for sub-process respectively. 

The \textbf{pc} is referenced within the produced \tlaplus translation as \[ pc[Process Id][SubProcess Index] \].
		  
The entire TLA+ translation of the two phase commit example can be found in appendix \ref{appendix:examples}, now we will be focusing on the translation of the process in figure \ref{2pcSub}.

\hfill\\
\begin{enumerate}

\item Generate the set $SubProcSet$ that identifies the sub-processes based on the process type.
In our example we have two process types, one for Agents and the other for Coordinators, both process types have two sub-processes.

\FloatBarrier
\begin{figure}[!h]
\label{2pcSubTla}
\begin{lstlisting}[caption = Distributed-PlusCal Translation- SubProcSet, frame = tlrb, firstnumber = 1]
SubProcSet == [n \in ProcSet |-> IF n \in Agent THEN 1..2
                             ELSE (**Coord**) 1..2]

\end{lstlisting}
\end{figure}
\FloatBarrier

\item Generate the set $pc$ variable to as a point of control for the processes and sub-processes.
In our example, for processes of type Agent we have two sub-processes. The first sub-process begins execution at action "a1", and the second one begins at action "a3".
\hfill\\

\FloatBarrier
\begin{figure}[!h]
\begin{lstlisting}[caption = TLA+ translation for Sub-Processes, frame = tlrb, firstnumber = 1]

Init == (* Global variables *)
        /\ coord = {}
        /\ agt = [a0 \in Agent |-> {}]
        (* Process a *)
        /\ aState = [self \in Agent |-> "unknown"]
        (* Process c *)
        /\ cState = "unknown"
        /\ commits = {}
        /\ msg = {}
        /\ pc = [self \in ProcSet |-> CASE self \in Agent -> <<"a1","a3">>
                                        [] self = Coord -> <<"c1","c2">>]
\end{lstlisting}
\end{figure}
\FloatBarrier

\item Translate the process labels into \tlaplus actions, verifying the pc variable at each action based on both the process identifier and the sub-process identifier.
In our example, when the $pc$ variable is accessed in <<"a1", "a2">> the sub-process ID is 1, and from <<"a3", "a4">> the the sub-process ID is 2.

\FloatBarrier
\begin{figure}
\begin{lstlisting}[caption = TLA+ translation for Sub-Processes, frame = tlrb, firstnumber = 1]

\* Process 1 Sub-Process 1 : an action with the statements in a1 label
a1(self) == /\ pc[self] [1] = "a1"
            /\ IF aState[self] = "unknown"
                  THEN /\ \E st \in {"accept", "refuse"}:
                            /\ aState' = [aState EXCEPT ![self] = st]
                            /\ coord' = (coord \cup {[type |-> st, agent |-> self]})
                  ELSE /\ TRUE
                       /\ UNCHANGED << coord, aState >>
            /\ pc' = [pc EXCEPT ![self] = [@  EXCEPT ![1] = "a2"]]
            /\ UNCHANGED << agt, cState, commits, msg >>

a2(self) == /\ pc[self] [1] = "a2"
            /\ (aState[self] \in {"commit", "abort"})
            /\ pc' = [pc EXCEPT ![self] = [@  EXCEPT ![1] = "Done"]]
            /\ UNCHANGED << coord, agt, aState, cState, commits, msg >>

a3(self) == /\ pc[self] [2] = "a3"
            /\ (aState[self] # "unknown")
            /\ \E a1519 \in agt[self]:
                 /\ aState' = [aState EXCEPT ![self] = a1519]
                 /\ agt' = [agt EXCEPT ![self] = agt[self] \ {a1519}]
            /\ pc' = [pc EXCEPT ![self] = [@  EXCEPT ![2] = "a4"]]
            /\ UNCHANGED << coord, cState, commits, msg >>

a4(self) == /\ pc[self] [2] = "a4"
            /\ agt' = [a0 \in Agent |-> {}]
            /\ pc' = [pc EXCEPT ![self] = [@  EXCEPT ![2] = "Done"]]
            /\ UNCHANGED << coord, aState, cState, commits, msg >>
\end{lstlisting}
\end{figure}
\FloatBarrier

\end{enumerate}


\section{Evaluation}

Distributed-PlusCal is a language that provides algorithm designers with an interface in which distributed algorithms and their properties can be expressed naturally.

While the PlusCal language may enforce some limitation that make it difficult to express distributed algorithms in a natural way, Distributed-PlusCal introduced constructs that can overcome these limitation.

In distributed algorithms several processes or threads may coexist and asynchronously communicate while being a part of a distributed node. Distributed-PlusCal firstly introduced Sub-processes \ref{subProcess} to allow the declaration of these threads, and secondly we provided the user with communication channel constructs \ref{comChannels} to assist in the communication between those processes.
\\\\
Extending the PlusCal translator introduced some limitations in our translation process;however, it also allowed our translator to be backward compatible. The Distributed-PlusCal translator can translate PlusCal models as well as Distributed-PlusCal models.

\chapter{Code Documentation}

\section{general structure of the toolbox and it's components}
try to describe the general flow

\section{parsing and expansion process}


\chapter{Conclusion and future work}


\begin{appendices}
\chapter{Distributed-PlusCal to TLA+ Examples}
\section{Two Phase Commit}
\label{appendix:examples}

\begin{lstlisting}[caption = TLA+ translation for Sub-Processes, frame = tlrb, firstnumber = 1]

----------------------------- MODULE 2pc -----------------------------
EXTENDS Sequences, Naturals

CONSTANTS Coord, Agent

State == {"unknown", "accept", "refuse", "commit", "abort"}

    
(* PlusCal options (-distpcal) *)

(***
--algorithm TPC {
 
  \* message channels
  channels coord, agt[Agent];
     
  fair process (a \in Agent)
  variable aState = "unknown"; {

a1: if (aState = "unknown") {
        with(st \in {"accept", "refuse"}) {
          aState := st;
          send(coord, [type |-> st, agent |-> self]);
        };
    };
    a2: await(aState \in {"commit", "abort"})
    
  } {
    
    a3:await (aState # "unknown");
       receive(agt[self], aState); 
       
    a4:clear(agt);
  }

  fair process (c = Coord) 
  variables cState = "unknown",
            commits = {}, msg = {};
             \* agents that agree to commit
  {
    c1: await(cState \in {"commit", "abort"});    
        broadcast(agt, [ag \in Agent|-> cState]);
  } {
        
     c2:while (cState \notin {"abort", "commit"}) {
        receive(coord, msg);
            if (msg.type = "refuse") {
                cState := "abort";
            }
            else if (msg.type = "accept") {
                commits := commits \cup {msg.agent};
                 if (commits = Agent) {
                    cState := "commit";
                 }
              }
        }
  }
 }
***)
\* BEGIN TRANSLATION
VARIABLES coord, agt, pc, aState, cState, commits, msg

vars == << coord, agt, pc, aState, cState, commits, msg >>

ProcSet == (Agent) \cup {Coord}

SubProcSet == [n \in ProcSet |-> IF n \in Agent THEN 1..2
                             ELSE (**Coord**) 1..2]

Init == (* Global variables *)
        /\ coord = {}
        /\ agt = [a0 \in Agent |-> {}]
        (* Node a *)
        /\ aState = [self \in Agent |-> "unknown"]
        (* Node c *)
        /\ cState = "unknown"
        /\ commits = {}
        /\ msg = {}
        /\ pc = [self \in ProcSet |-> CASE self \in Agent -> <<"a1","a3">>
                                        [] self = Coord -> <<"c1","c2">>]

a1(self) == /\ pc[self] [1] = "a1"
            /\ IF aState[self] = "unknown"
                  THEN /\ \E st \in {"accept", "refuse"}:
                            /\ aState' = [aState EXCEPT ![self] = st]
                            /\ coord' = (coord \cup {[type |-> st, agent |-> self]})
                  ELSE /\ TRUE
                       /\ UNCHANGED << coord, aState >>
            /\ pc' = [pc EXCEPT ![self] = [@  EXCEPT ![1] = "a2"]]
            /\ UNCHANGED << agt, cState, commits, msg >>

a2(self) == /\ pc[self] [1] = "a2"
            /\ (aState[self] \in {"commit", "abort"})
            /\ pc' = [pc EXCEPT ![self] = [@  EXCEPT ![1] = "Done"]]
            /\ UNCHANGED << coord, agt, aState, cState, commits, msg >>

a3(self) == /\ pc[self] [2] = "a3"
            /\ (aState[self] # "unknown")
            /\ \E a1519 \in agt[self]:
                 /\ aState' = [aState EXCEPT ![self] = a1519]
                 /\ agt' = [agt EXCEPT ![self] = agt[self] \ {a1519}]
            /\ pc' = [pc EXCEPT ![self] = [@  EXCEPT ![2] = "a4"]]
            /\ UNCHANGED << coord, cState, commits, msg >>

a4(self) == /\ pc[self] [2] = "a4"
            /\ agt' = [a0 \in Agent |-> {}]
            /\ pc' = [pc EXCEPT ![self] = [@  EXCEPT ![2] = "Done"]]
            /\ UNCHANGED << coord, aState, cState, commits, msg >>

a(self) == a1(self) \/ a2(self) \/ a3(self) \/ a4(self)

c1 == /\ pc[Coord] [1] = "c1"
      /\ (cState \in {"commit", "abort"})
      /\ agt' = [ag \in Agent |-> agt[ag] \cup  {cState} ]
      /\ pc' = [pc EXCEPT ![Coord] = [@  EXCEPT ![1] = "Done"]]
      /\ UNCHANGED << coord, aState, cState, commits, msg >>

c2 == /\ pc[Coord] [2] = "c2"
      /\ IF cState \notin {"abort", "commit"}
            THEN /\ \E c1512 \in coord:
                      /\ coord' = coord \ {c1512}
                      /\ msg' = c1512
                 /\ IF msg'.type = "refuse"
                       THEN /\ cState' = "abort"
                            /\ UNCHANGED commits
                       ELSE /\ IF msg'.type = "accept"
                                  THEN /\ commits' = (commits \cup {msg'.agent})
                                       /\ IF commits' = Agent
                                             THEN /\ cState' = "commit"
                                             ELSE /\ TRUE
                                                  /\ UNCHANGED cState
                                  ELSE /\ TRUE
                                       /\ UNCHANGED << cState, commits >>
                 /\ pc' = [pc EXCEPT ![Coord] = [@  EXCEPT ![2] = "c2"]]
            ELSE /\ pc' = [pc EXCEPT ![Coord] = [@  EXCEPT ![2] = "Done"]]
                 /\ UNCHANGED << coord, cState, commits, msg >>
      /\ UNCHANGED << agt, aState >>

c == c1 \/ c2

(* Allow infinite stuttering to prevent deadlock on termination. *)
Terminating == /\ \A self \in ProcSet : \A sub \in SubProcSet[self]: pc[self][sub] = "Done"
               /\ UNCHANGED vars

Next == c
           \/ (\E self \in Agent: a(self))
           \/ Terminating

Spec == /\ Init /\ [][Next]_vars
        /\ \A self \in Agent : WF_vars(a(self))
        /\ WF_vars(c)

Termination == <>(\A self \in ProcSet: \A sub \in SubProcSet[self] : pc[self][sub] = "Done")

\* END TRANSLATION
=============================================================================

\end{lstlisting}

In the translation above, every label is transformed into an action, some actions are also created when needed, for example the receive function and the clear function in process 1 sub-process 2 both modify the same channel and it would be wrong to place the two assignments in the same atomic step thus an auxiliary action is created to hold the body of the clear function.

- insert labels in example, before await and also for receive and clear then modify the example


\section{Lamport Mutex}

\end{appendices}

\bibliographystyle{alpha}
\bibliography{report}

\end{document}
