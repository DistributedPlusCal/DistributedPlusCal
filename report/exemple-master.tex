% Exemple d'utilisation de la classe `thesul' pour un master.
% D. Roegel, 3/4/2013
%
% Note : les couvertures de Master ne sont pas finalisées dans thesul.
%        Indiquez-moi ce qu'il faut mettre.
%
\documentclass{thesul}




%-------------------------------------------------------------------
%                         Les références
%-------------------------------------------------------------------

\NoChapterNumberInRef
\NoChapterPrefix


\begin{document}

      \OddHead={{\leftmark\rightmark}{\hfil\slshape\rightmark}}
      \EvenHead={{\leftmark}{{\slshape\leftmark}\hfil}}
      \OddFoot={\hfil\thepage}
      \EvenFoot={\thepage\hfil}
      \pagestyle{ThesisHeadingsII}

%-------------------------------------------------------------------
%            Réinitialisation de la numérotation des chapitres
%-------------------------------------------------------------------

% Si la commande suivante est présente,
% elle doit figurer APRÈS \begin{document}
% et avant la première commande \part
\ResetChaptersAtParts 





\MasterUL
\ThesisTitle{Formal Verification of Distributed Algorithms using Distributed-PlusCal}
\ThesisAuthor{Heba Al-kayed}
\ThesisPresentedThe{soutenu le 23 septembre 2016}
\President    = {Le pr\'esident}
\Rapporteurs  = {Le rapporteur 1\\
                 Le rapporteur 2\\
                 Le rapporteur 3}
\Examinateurs = {L'examinateur 1\\
                 L'examinateur 2}

\MakeThesisTitlePage


%-------------------------------------------------------------------
%                          remerciements
%-------------------------------------------------------------------

%\DontFrameThisInToc
\begin{ThesisAcknowledgments}
Les remerciements.
\end{ThesisAcknowledgments}

%-------------------------------------------------------------------
%                            dédicace
%-------------------------------------------------------------------

\begin{ThesisDedication}
Je dédie cette thèse\\
à ma machine.\\
Oui, à Pandore,\\
qui fut la première de toutes.
\end{ThesisDedication}


%-------------------------------------------------------------------
%                  écriture de `Chapitre' et `Partie' 
%                      dans la table des matières
%-------------------------------------------------------------------

\WritePartLabelInToc
\WriteChapterLabelInToc

%-------------------------------------------------------------------
%                        table des matières
%-------------------------------------------------------------------

\tableofcontents

% Pour ne pas avoir le mot « Chapitre » au début de chaque chapitre.
\NoChapterHead


% La commande \mainmatter permet de passer
% à la numérotation arabe (ce que fait \pagenumbering{arabic}) 
% et de faire commencer la nouvelle page 1 sur une page impaire.
% On évitera donc d'utiliser directement \pagenumbering{arabic}.
\mainmatter


\chapter{Introduction}
%%============================
This is a very nice thesis about TLA~\cite{tlabook}.

\paragraph*{Motivations}
why this extension

\paragraph*{Outline}


\chapter{Background info}
%%============================

a brief overview of ModelChecking, TLA and Pluscal possibly with an example to show why it's used or its advantages.
maybe mention real life applications like amazon's AWS.


\section{TLA+}
%%============================

\cite{tlabook}.

\section{PlusCal algorithm language}
%%============================
 

\chapter{Related work}
%%============================
position of our work compared with other work

\section{PGO}

\subsection{Modular PlusCal}

\chapter{Distributed PlusCal}

\section{Components}

\subsection{Nodes}

an example to show the nodes/threads.
\subsection{Channels}
\paragraph{Unordered channels}
example with it's translation
\paragraph{FIFO channels}
example with it's translation
\paragraph{Supported channel functions}
expected syntax and limitations
\section{Examples}
our examples with their translations

\chapter{Code Documentation}

\section{general structure of the toolbox and it's components}
try to describe the general flow

\section{parsing and expansion process}

\section{some software-based diagram}

or maybe an AST description graph

\chapter{Conclusion and future work}
\bibliographystyle{alpha}
\bibliography{report}

\end{document}
