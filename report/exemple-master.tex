% Exemple d'utilisation de la classe `thesul' pour un master.
% D. Roegel, 3/4/2013
%
% Note : les couvertures de Master ne sont pas finalisées dans thesul.
%        Indiquez-moi ce qu'il faut mettre.
%
\documentclass{thesul}




%-------------------------------------------------------------------
%                         Les références
%-------------------------------------------------------------------

\NoChapterNumberInRef
\NoChapterPrefix


\begin{document}

      \OddHead={{\leftmark\rightmark}{\hfil\slshape\rightmark}}
      \EvenHead={{\leftmark}{{\slshape\leftmark}\hfil}}
      \OddFoot={\hfil\thepage}
      \EvenFoot={\thepage\hfil}
      \pagestyle{ThesisHeadingsII}

%-------------------------------------------------------------------
%            Réinitialisation de la numérotation des chapitres
%-------------------------------------------------------------------

% Si la commande suivante est présente,
% elle doit figurer APRÈS \begin{document}
% et avant la première commande \part
\ResetChaptersAtParts 





\MasterUL
\ThesisTitle{Une première étude}
\ThesisAuthor{Toto}
\ThesisPresentedThe{soutenu le 23 septembre 2016}
\President    = {Le pr\'esident}
\Rapporteurs  = {Le rapporteur 1\\
                 Le rapporteur 2\\
                 Le rapporteur 3}
\Examinateurs = {L'examinateur 1\\
                 L'examinateur 2}

\MakeThesisTitlePage


%-------------------------------------------------------------------
%                          remerciements
%-------------------------------------------------------------------

%\DontFrameThisInToc
\begin{ThesisAcknowledgments}
Les remerciements.
\end{ThesisAcknowledgments}

%-------------------------------------------------------------------
%                            dédicace
%-------------------------------------------------------------------

\begin{ThesisDedication}
Je dédie cette thèse\\
à ma machine.\\
Oui, à Pandore,\\
qui fut la première de toutes.
\end{ThesisDedication}


%-------------------------------------------------------------------
%                  écriture de `Chapitre' et `Partie' 
%                      dans la table des matières
%-------------------------------------------------------------------

\WritePartLabelInToc
\WriteChapterLabelInToc

%-------------------------------------------------------------------
%                        table des matières
%-------------------------------------------------------------------

\tableofcontents

% Pour ne pas avoir le mot « Chapitre » au début de chaque chapitre.
\NoChapterHead


% La commande \mainmatter permet de passer
% à la numérotation arabe (ce que fait \pagenumbering{arabic}) 
% et de faire commencer la nouvelle page 1 sur une page impaire.
% On évitera donc d'utiliser directement \pagenumbering{arabic}.
\mainmatter


\chapter{Introduction}
%%============================
This is a very nice thesis about TLA~\cite{tlabook}.

\paragraph*{Motivations}

tttttt rrrrr tttttt rrrrr tttttt rrrrr tttttt rrrrr tttttt rrrrr
tttttt rrrrr tttttt rrrrr tttttt rrrrr tttttt rrrrr tttttt rrrrr
tttttt rrrrr tttttt rrrrr tttttt rrrrr tttttt rrrrr tttttt rrrrr
tttttt rrrrr tttttt rrrrr tttttt rrrrr tttttt rrrrr tttttt rrrrr

\paragraph*{Outline}

tttttt rrrrr tttttt rrrrr tttttt rrrrr tttttt rrrrr tttttt rrrrr
tttttt rrrrr tttttt rrrrr tttttt rrrrr tttttt rrrrr tttttt rrrrr
tttttt rrrrr tttttt rrrrr tttttt rrrrr tttttt rrrrr tttttt rrrrr
tttttt rrrrr tttttt rrrrr tttttt rrrrr tttttt rrrrr tttttt rrrrr

\chapter{Translation}
%%============================

\section{This is a section}
%%============================

aaaa

\subsection{This is a subsection}
%%============================
aaaaa bbbbb ccccc aaaaa bbbbb ccccc aaaaa bbbbb ccccc aaaaa bbbbb
ccccc aaaaa bbbbb ccccc aaaaa bbbbb ccccc aaaaa bbbbb ccccc aaaaa
bbbbb ccccc aaaaa bbbbb ccccc aaaaa bbbbb ccccc aaaaa bbbbb ccccc
aaaaa bbbbb ccccc aaaaa bbbbb ccccc aaaaa bbbbb ccccc aaaaa bbbbb
ccccc aaaaa bbbbb ccccc aaaaa bbbbb ccccc aaaaa bbbbb ccccc aaaaa
bbbbb ccccc aaaaa bbbbb ccccc aaaaa bbbbb ccccc aaaaa bbbbb ccccc
aaaaa bbbbb ccccc aaaaa bbbbb ccccc aaaaa bbbbb ccccc aaaaa bbbbb
ccccc aaaaa bbbbb ccccc aaaaa bbbbb ccccc aaaaa bbbbb ccccc aaaaa
bbbbb ccccc aaaaa bbbbb ccccc aaaaa bbbbb ccccc aaaaa bbbbb ccccc
aaaaa bbbbb ccccc aaaaa bbbbb ccccc aaaaa bbbbb ccccc

\subsection{This is another subsection}
%%============================
aaaaa bbbbb ccccc aaaaa bbbbb ccccc aaaaa bbbbb ccccc aaaaa bbbbb
ccccc aaaaa bbbbb ccccc aaaaa bbbbb ccccc aaaaa bbbbb ccccc aaaaa
bbbbb ccccc aaaaa bbbbb ccccc aaaaa bbbbb ccccc aaaaa bbbbb ccccc
aaaaa bbbbb ccccc aaaaa bbbbb ccccc aaaaa bbbbb ccccc aaaaa bbbbb
ccccc aaaaa bbbbb ccccc aaaaa bbbbb ccccc aaaaa bbbbb ccccc aaaaa
bbbbb ccccc aaaaa bbbbb ccccc aaaaa bbbbb ccccc aaaaa bbbbb ccccc
aaaaa bbbbb ccccc aaaaa bbbbb ccccc aaaaa bbbbb ccccc aaaaa bbbbb
ccccc aaaaa bbbbb ccccc aaaaa bbbbb ccccc aaaaa bbbbb ccccc aaaaa
bbbbb ccccc aaaaa bbbbb ccccc aaaaa bbbbb ccccc aaaaa bbbbb ccccc
aaaaa bbbbb ccccc aaaaa bbbbb ccccc aaaaa bbbbb ccccc

\section{This is another section}
%%============================

\bibliographystyle{alpha}
\bibliography{report}

\end{document}
