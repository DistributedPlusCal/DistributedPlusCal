\documentclass{beamer}
\usecolortheme{orchid}


\usepackage{capt-of}
\usepackage{placeins}
\usepackage[toc,page]{appendix}
\usepackage{multirow}
\usepackage{xspace}
\usepackage{latexsym}
\usepackage{amssymb}
\usepackage{listings}
\usepackage{tikz}

\usepackage{color}
\usepackage[T1]{fontenc}
\usepackage{pcallst}

\definecolor{dkgreen}{rgb}{0,0.6,0}
\definecolor{gray}{rgb}{0.5,0.5,0.5}
\definecolor{mauve}{rgb}{0.58,0,0.82}
\definecolor{orange}{rgb}{1.0,0.49,0.0}

\usepackage{pifont}

\lstset{frame=single,
  aboveskip=3mm,
  belowskip=3mm,
  showstringspaces=false,
  basicstyle={\tiny\ttfamily},
  numbers=left,
  stepnumber=1,
  numberstyle=\tiny\color{gray},
  commentstyle=\color{dkgreen},
  breaklines=true,
  breakatwhitespace=true,
  tabsize=1,
  morecomment=[l]{\\*},
  morekeywords={}
}

\newcommand\tab[1][1cm]{\hspace*{#1}}
\renewcommand{\floatpagefraction}{.99}
\renewcommand{\textfraction}{.01}

\newcommand{\tlaplus}{TLA\textsuperscript{+}\xspace}
\newcommand{\EXCEPT}{\textsc{except}}
\newcommand{\IF}{\textsc{if}}
\newcommand{\THEN}{\textsc{then}}
\newcommand{\ELSE}{\textsc{else}}
\newcommand{\seq}[1]{\langle #1 \rangle}

\newcommand{\keyword}[1]{\textbf{#1}}
\newcommand{\entity}[1]{\ensuremath{\langle}#1\ensuremath{\rangle}}

% editorial comments in the text or in marginal notes
% 1st argument: initials of the person making the comment,
% 2nd argument: comment to insert
\long\def\ednote#1#2{\par\noindent\framebox{\begin{minipage}{0.99\linewidth}\linespread{.7}\footnotesize #1: #2\end{minipage}}\par}
\newcommand{\edmargin}[2]{\marginpar{\raggedright\linespread{.7}\footnotesize #1: #2}}

%% make TeX use text italic font in math environments
\makeatletter
\newcounter{abr@ctr}
\newcommand{\abr@c}{\c@abr@ctr\advance\c@abr@ctr\@ne}

\DeclareSymbolFont{tlaitalics}{\encodingdefault}{cmr}{m}{it}
\let\itfam\symtlaitalics

\newcommand{\noTeXmath}{%
\c@abr@ctr=\itfam
\multiply\c@abr@ctr"100\relax
\advance\c@abr@ctr "7061\relax
\mathcode`a=\abr@c\mathcode`b=\abr@c\mathcode`c=\abr@c\mathcode`d=\abr@c
\mathcode`e=\abr@c\mathcode`f=\abr@c\mathcode`g=\abr@c\mathcode`h=\abr@c
\mathcode`i=\abr@c\mathcode`j=\abr@c\mathcode`k=\abr@c\mathcode`l=\abr@c
\mathcode`m=\abr@c\mathcode`n=\abr@c\mathcode`o=\abr@c\mathcode`p=\abr@c
\mathcode`q=\abr@c\mathcode`r=\abr@c\mathcode`s=\abr@c\mathcode`t=\abr@c
\mathcode`u=\abr@c\mathcode`v=\abr@c\mathcode`w=\abr@c\mathcode`x=\abr@c
\mathcode`y=\abr@c\mathcode`z=\abr@c
\c@abr@ctr=\itfam
\multiply\c@abr@ctr"100\relax
\advance\c@abr@ctr "7041\relax
\mathcode`A=\abr@c\mathcode`B=\abr@c\mathcode`C=\abr@c\mathcode`D=\abr@c
\mathcode`E=\abr@c\mathcode`F=\abr@c\mathcode`G=\abr@c\mathcode`H=\abr@c
\mathcode`I=\abr@c\mathcode`J=\abr@c\mathcode`K=\abr@c\mathcode`L=\abr@c
\mathcode`M=\abr@c\mathcode`N=\abr@c\mathcode`O=\abr@c\mathcode`P=\abr@c
\mathcode`Q=\abr@c\mathcode`R=\abr@c\mathcode`S=\abr@c\mathcode`T=\abr@c
\mathcode`U=\abr@c\mathcode`V=\abr@c\mathcode`W=\abr@c\mathcode`X=\abr@c
\mathcode`Y=\abr@c\mathcode`Z=\abr@c}

\newcommand{\TeXmath}{%
\mathcode`a="7161\mathcode`b="7162\mathcode`c="7163\mathcode`d="7164%
\mathcode`e="7165\mathcode`f="7166\mathcode`g="7167\mathcode`h="7168%
\mathcode`i="7169\mathcode`j="716A\mathcode`k="716B\mathcode`l="716C%
\mathcode`m="716D\mathcode`n="716E\mathcode`o="716F\mathcode`p="7170%
\mathcode`q="7171\mathcode`r="7172\mathcode`s="7173\mathcode`t="7174%
\mathcode`u="7175\mathcode`v="7176\mathcode`w="7177\mathcode`x="7178%
\mathcode`y="7179\mathcode`z="717A\mathcode`A="7141\mathcode`B="7142%
\mathcode`C="7143\mathcode`D="7144\mathcode`E="7145\mathcode`F="7146%
\mathcode`G="7147\mathcode`H="7148\mathcode`I="7149\mathcode`J="714A%
\mathcode`K="714B\mathcode`L="714C\mathcode`M="714D\mathcode`N="714E%
\mathcode`O="714F\mathcode`P="7150\mathcode`Q="7151\mathcode`R="7152%
\mathcode`S="7153\mathcode`T="7154\mathcode`U="7155\mathcode`V="7156%
\mathcode`W="7157\mathcode`X="7158\mathcode`Y="7159\mathcode`Z="715A}

\tikzstyle{startstop} = [rectangle, rounded corners, minimum width=3cm, minimum height=1cm,text centered, draw=black, fill=red!30]

\tikzstyle{io} = [trapezium, trapezium left angle=70, trapezium right angle=110, minimum width=3cm, minimum height=1cm, text centered, draw=black, fill=blue!30]

\tikzstyle{process} = [rectangle, minimum width=3cm, minimum height=1cm, text centered, draw=black, fill=orange!30]

\tikzstyle{decision} = [diamond, minimum width=3cm, minimum height=1cm, text centered, draw=black, fill=green!30]

\tikzstyle{arrow} = [thick,->,>=stealth]

\noTeXmath
\makeatother

\makeatletter
\newcommand*{\centerfloat}{%
  \parindent \z@
  \leftskip \z@ \@plus 1fil \@minus \textwidth
  \rightskip\leftskip
  \parfillskip \z@skip}
\makeatother

\mode<presentation>
{
  \usetheme{default}      % or try Darmstadt, Madrid, Warsaw, ...
  \usecolortheme{default} % or try albatross, beaver, crane, ...
  \usefonttheme{default}  % or try serif, structurebold, ...
  \setbeamertemplate{navigation symbols}{}
  \setbeamertemplate{caption}[numbered]
} 

\usepackage[english]{babel}
\usepackage[utf8]{inputenc}
\usepackage[T1]{fontenc}

\title[Your Short Title]{An Extension of PlusCal for Modeling
Distributed Algorithms}
%\subtitle{Master of Computer Science - MFLS}
\institute{University of Lorraine, CNRS, Inria, Nancy, France}
\author[]{Heba Alkayed, Horatiu Cirstea, Stephan Merz}


%\date{6th July 2020}

\begin{document}

\begin{frame}
  \titlepage
\end{frame}

% Uncomment these lines for an automatically generated outline.
%\begin{frame}{Outline}
%  \tableofcontents
%\end{frame}

\section{Introduction}

\begin{frame}{Introduction}

\begin{block}{Formal methods for distributed algorithms}
\begin{itemize}

  \item Algorithms modeled using \tlaplus can be formally verified using the \tlaplus Toolbox
 \item PlusCal algorithms have a more familiar syntax and can be translated to \tlaplus 
\end{itemize}
\end{block}
\vskip 1cm

\end{frame}

\section{Distributed PlusCal}

\begin{frame}[fragile]{Distributed PlusCal Algorithms}

\begin{itemize}
  \item An extension of PlusCal offering constructs for modeling distributed algorithms 
  \item Can be translated into a TLA+ specification
  \item Introduces
  \begin{itemize} 
        \item Sub-processes
        \item Communication channels
    \end{itemize}
\end{itemize}

\end{frame}


\begin{frame}[fragile]{General Structure of an algorithm}
\begin{lstlisting}[language=pluscal, frame = tlrb, numbers = none]
(* --algorithm <<@\textcolor{black}{algorithm}@> name>
(* Declaration section *)
variables <variable declarations>
<@\textcolor{blue}{channels}@> <channel declarations>
<@\textcolor{blue}{fifos}@> <fifo declarations>
(* ... *)
(* Processes section *)
process (<name> [=|\in] <Expr>))
  variables <variable declarations>
  <@\textcolor{blue}{<subprocesses>}@>
*)
\end{lstlisting}
\end{frame}

\section{Distributed PlusCal Features}

\subsection{Communication Channels}

\begin{frame}[fragile]{Communication Channels}
    \begin{itemize}
        \item Classified by the way they handle the addition and removal of messages 
            \begin{itemize}
                \item Unordered channels 
                \item FIFO channels 
            \end{itemize}
     \item Supported operators 
            \begin{itemize}
                \item \verb|send(ch, el)|
                \item \verb|receive(ch, var)|
                \item \verb|broadcast(ch, [x \in S \mapsto e(x)]| 
                \item \verb|multicast(ch, [x \in S \mapsto e(x)]|
                \item \verb|clear(ch)|
            \end{itemize}

    \end{itemize}
\end{frame}

\begin{frame}[fragile]{Unordered Channels}
    \begin{itemize}
        \item \keyword{channel} or \keyword{channels}, as shown below.
        \[
            \keyword{channel}\ \entity{id}[\entity{Expr_1},\dots,\entity{Expr_N}];
        \]
         \item based on \tlaplus sets
     \item Operator translation to \tlaplus
            \begin{itemize}
                \item \verb|send(ch, el)| $\triangleq$
                \verb|chan' = chan \cup {el}| 
                \newline
                \item \verb|receive(chan, var)| $\triangleq$
                      \verb|\E e \in chan:| \newline \tab\tab\tab\tab
                         \verb|  /\ var' = e|\newline \tab\tab\tab\tab
                         \verb|  /\ chan' =  chan \ {e}|
            \end{itemize}

    \end{itemize}
\end{frame}

\begin{frame}[fragile]{FIFO Channels}
    \begin{itemize}
        \item \keyword{fifo} or \keyword{fifos}, as shown below.
        \[
            \keyword{fifo}\ \entity{id}[\entity{Expr_1},\dots,\entity{Expr_N}];
        \]
         \item based on \tlaplus sequences
     \item Operator translation to \tlaplus

            \begin{itemize}
                \item \verb|send(ch, el)| $\triangleq$
                \verb|chan' = Append(chan, el)| 
                \newline
                \item \verb|receive(chan, var)| $\triangleq$
                      \verb| /\ Len(chan) > 0 | \newline \tab\tab\tab\tab
                         \verb|/\ var' = [Head(chan)]|\newline \tab\tab\tab\tab
                         \verb|/\ chan' =  Tail(chan)|
            \end{itemize}
    \end{itemize}
\end{frame}

\subsection{Sub-Processes}

\begin{frame}[fragile]{Sub-Processes}
    \begin{itemize}
     \item Enables the process to execute multiple tasks in parallel.
     \item Each sub-process consists of labeled statements.
     \item The special variable pc was modified to have the following definition

\[
pc = [self \in ProcSet \mapsto [self \in IdSet \mapsto \seq{"lbl",\dots}]]
\]	
    where \verb|ProcSet| is a set that contains all the process identifiers, and the labels that appear in the sequence are the entry point actions for each sub-process.

    \end{itemize}
\end{frame}

\section{Example}
\begin{frame}[fragile]{Distributed PlusCal Example}
 \begin{lstlisting}[language=pluscal, frame = tlrb, numbers=none]  
 process(n \in Nodes)
     variables clock = 0, req = [n \in Nodes |-> 0],
               ack = {}, sndr, msg;
   { \* thread executing the main algorithm
ncs: while (<@\textcolor{blue}{TRUE}@>) {
       skip;  \* non-critical section
try:   clock := clock + 1; req[self] := clock; ack := {self};
       multicast(network, [self, nd \in Nodes |-> Request(clock)]);
enter: await (ack = Nodes /\ \A n \in Nodes \ {self} : beats(self, n));
cs:    skip;  \* critical section
exit:  clock := clock + 1;
       multicast(network, [self, n \in Nodes \ {self} |->
                           Release(clock)]);
     } \* end while
  } 
\end{lstlisting}
\end{frame}

\begin{frame}[fragile]{Translation to TLA+}
 \begin{lstlisting}[language=pluscal, frame = tlrb, numbers=none]  
 exit:  
    clock := clock + 1;
    multicast(network, [self, n \in Nodes \ {self} |->
                              Release(clock)]);
\end{lstlisting}


\begin{lstlisting}[language=pluscal, frame = tlrb, numbers=none]  
exit(self) == 
    @/\ pc[self] [1] = "exit"@
\end{lstlisting}
\end{frame}

\begin{frame}[fragile]{Translation to TLA+}
 \begin{lstlisting}[language=pluscal, frame = tlrb, numbers=none]  
 exit:  
    @clock := clock + 1;}@
    multicast(network, [self, n \in Nodes \ {self} |->
                              Release(clock)]);
\end{lstlisting}


\begin{lstlisting}[language=pluscal, frame = tlrb, numbers=none]  
exit(self) == 
    /\ pc[self] [1] = "exit"
    @/\ clock' = [clock EXCEPT ![self] = clock[self] + 1]}@
\end{lstlisting}
\end{frame}


\begin{frame}[fragile]{Translation to TLA+}
 \begin{lstlisting}[language=pluscal, frame = tlrb, numbers=none]  
 exit:  
    clock := clock + 1;}
    @multicast(network, [self, n \in Nodes \ {self} |->                 Release(clock)]);@
\end{lstlisting}


\begin{lstlisting}[language=pluscal, frame = tlrb, numbers=none]  
exit(self) == 
    /\ pc[self] [1] = "exit"
    /\ clock' = [clock <@\textcolor{violet}{EXCEPT}@> ![self] = clock[self] + 1]
    @/\ network' = [<<slf, n>> \in DOMAIN network |->
        IF
            slf = self /\ n \in Nodes \ { self } 
        THEN
            Append(network[slf, n], Release(clock'[self])) 
        ELSE
            network[slf, n]]@
\end{lstlisting}
\end{frame}

\begin{frame}[fragile]{Translation to TLA+}
 \begin{lstlisting}[language=pluscal, frame = tlrb, numbers=none]  
 exit:  
    clock := clock + 1;
    multicast(network, [self, n \in Nodes \ {self} |->
                              Release(clock)]);
\end{lstlisting}


\begin{lstlisting}[language=pluscal, frame = tlrb, numbers=none]  
exit(self) == 
    /\ pc[self] [1] = "exit"
    /\ clock' = [clock <@\textcolor{violet}{EXCEPT}@> ![self] = clock[self] + 1]
    /\ network' = [<<slf, n>> \in <@\textcolor{violet}{DOMAIN}@> network |->
        <@\textcolor{violet}{IF}@> 
            slf = self /\ n \in Nodes \ { self } 
        <@\textcolor{violet}{THEN}@> 
            Append(network[slf, n], Release(clock'[self])) 
        <@\textcolor{violet}{ELSE}@>
            network[slf, n]]
    @/\ pc' = [pc EXCEPT ![self][1] = "ncs"]@
\end{lstlisting}
\end{frame}

\begin{frame}[fragile]{Translation to TLA+}
 \begin{lstlisting}[language=pluscal, frame = tlrb, numbers=none]  
 exit:  
    clock := clock + 1;
    multicast(network, [self, n \in Nodes \ {self} |->
                              Release(clock)]);
\end{lstlisting}


\begin{lstlisting}[language=pluscal, frame = tlrb, numbers=none]  
exit(self) == 
    /\ pc[self] [1] = "exit"
    /\ clock' = [clock <@\textcolor{violet}{EXCEPT}@> ![self] = clock[self] + 1]
    /\ network' = [<<slf, n>> \in <@\textcolor{violet}{DOMAIN}@> network |->
        <@\textcolor{violet}{IF}@> 
            slf = self /\ n \in Nodes \ { self } 
        <@\textcolor{violet}{THEN}@> 
            Append(network[slf, n], Release(clock'[self])) 
        <@\textcolor{violet}{ELSE}@>
            network[slf, n]]
    /\ pc' = [pc <@\textcolor{violet}{EXCEPT}@> ![self][1] = "ncs"]
    /\ <@\textcolor{violet}{UNCHANGED}@> << req, ack, sndr, msg >>
\end{lstlisting}
\end{frame}



\begin{frame}{Contributions and future work}
\begin{block}{Contributions}
\begin{itemize}
    \item An extension of PlusCal called Distributed PlusCal
    \item Distributed PlusCal offers constructs that are designed for modeling distributed algorithms

    \item A backward compatible translator that translates from Distributed PlusCal and PlusCal  to TLA+


\end{itemize}
\end{block}
\begin{block}{Future Work}
In the future we aim to introduce more types of communication channels and channel operators.
\end{block}

\end{frame}

\end{document}